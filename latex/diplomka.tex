%!TEX TS-program = xelatex
%!TEX encoding = UTF-8 Unicode

\documentclass[12pt]{report}

\usepackage{tocloft}

%\setlength\cftparskip{-2pt}
%\setlength\cftbeforechapskip{0pt}

\usepackage{xecyr}
\usepackage[utf8x]{inputenc}
\usepackage[russian]{babel}

%\usepackage[utf8x]{inputenc}
%\usepackage[english,russian]{babel}

\usepackage[top=2.5cm, bottom=2.5cm, left=3cm, right=2.5cm]{geometry}                % See geometry.pdf to learn the layout options. There are lots.
\geometry{a4paper}                   % ... or a4paper or a5paper or ... 
%\geometry{landscape}                % Activate for for rotated page geometry
%\usepackage[parfill]{parskip}    % Activate to begin paragraphs with an empty line rather than an indent
\usepackage{graphicx}
%\usepackage{amssymb}

\usepackage{fancyhdr}
\pagestyle{fancy}
\lhead{}
\rhead{\footnotesize{Система слежения за лицом человека}}
\cfoot{}
\rfoot{\thepage}

\pagenumbering{arabic}

\usepackage{setspace}
\onehalfspacing

\usepackage{fontspec,xltxtra,xunicode}
\defaultfontfeatures{Mapping=tex-text}
\setromanfont[Mapping=tex-text]{Times New Roman}
\setsansfont[Scale=MatchLowercase,Mapping=tex-text]{Gill Sans}
\setmonofont[Scale=MatchLowercase]{Andale Mono}

% Will Robertson's fontspec.sty can be used to simplify font choices.
% To experiment, open /Applications/Font Book to examine the fonts provided on Mac OS X,
% and change "Hoefler Text" to any of these choices.


\setlength{\parindent}{0cm}
\setlength{\parskip}{12pt plus 0.5ex minus 0.2ex}
\renewcommand{\headrulewidth}{0pt}

%\title{Искусственный интеллект в играх на примере игры "Войны планет"}
%\author{Александр Мочёнов}
%\date{26 Октября 2010 г.}                                           
% Activate to display a given date or no date

\usepackage{titlesec}
\titleformat{\chapter}[hang]{\bf\normalsize\uppercase}{\thechapter}{2pc}{}
\titleformat{\section}[hang]{\bf\normalsize}{\thesection}{2pc}{}	
\titleformat{\subsection}[hang]{\bf\normalsize}{\thesubsection}{2pc}{}
\titlespacing{\chapter}{0pt}{*4}{*1}

\usepackage{hyperref} 
%\usepackage{graphicx}
\usepackage{pstricks}
%\DeclareGraphicsExtensions{.png}

%\newcommand{\executeiffilenewer}[3]{%
% \ifnum\pdfstrcmp{\pdffilemoddate{#1}}%
% {\pdffilemoddate{#2}} >0 {\immediate\write18{#3}}\fi%
%}
%
%\newcommand{\includesvg}[1]{%
%\executeiffilenewer{#1.svg}{#1.pdf}%
%{inkscape -z -D --file=#1.svg --export-pdf=#1.pdf --export-latex}%
%\input{#1.tex}%
%}
\usepackage{subfig}

\usepackage{url}

\usepackage{natbib}
\bibpunct{(}{)}{;}{a}{,}{,}

\newenvironment{myItemize}{
	\begin{itemize}
  		\setlength{\itemsep}{1pt}
  		\setlength{\parskip}{0pt}
  		\setlength{\parsep}{0pt}
}{\end{itemize}}

\newenvironment{myEnumerate}{
	\begin{enumerate}
  		\setlength{\itemsep}{1pt}
  		\setlength{\parskip}{0pt}
  		\setlength{\parsep}{0pt}
}{\end{enumerate}}

\usepackage{totcount}
\regtotcounter{figure}
\regtotcounter{table}
\regtotcounter{chapter}

\usepackage[acronym,toc]{glossaries}
\makeglossaries
%\input{glossary}


\renewcommand{\cfttoctitlefont}{\normalsize\textbf}
\setlength{\cftaftertoctitleskip}{18pt}


\usepackage{listings}


\begin{document}
%\renewcommand{\chaptername}{} 
%\renewcommand{\thechapter}{} 
%\renewcommand{\thesection}{\arabic{section}} 

\begin{titlepage}
  \begin{center}
	\uppercase{Высшая школа майнор}\\*
	Институт инфотехнологии\\*
	Веб программирование\\[8cm]
	Александр Мочёнов\\*
	IT-3-Q-V-Tal\\[0.5cm]
	\large
	\textbf{Система слежения за положением лица человека (на основе нейронной сети? (и облостей с кожным покровом?))}\\[1cm]
	\normalsize
	Дипломная работа\\[2cm]
	\begin{flushright}
		Руководитель:Jelena Faronova, MSc\\[7cm]
	\end{flushright}
	Таллинн 2010
  \end{center}
\end{titlepage}

\tableofcontents{\thispagestyle{fancyplain}}

\chapter*{Резюме}
\addcontentsline{toc}{chapter}{Резюме}
\thispagestyle{fancy}

TODO:

%\printglossary[title=Терминалогия и переводы,style=fancy,toctitle=Терминалогия и переводы]

\chapter*{Введение}
\addcontentsline{toc}{chapter}{Введение}
\thispagestyle{fancy}

Роботы в различных вариациях являются частью жизни человека. Робототехника уже
давно применяются, например, в индустриальном производстве, в детских игрушках,
авиации и многих других местах. Так же роботы применяются в военными (беспилотные
самолёты, роботы-сапёры), медицине и даже в космосе\footnote{\url{http://robonaut.jsc.nasa.gov/default.asp}}.

Тем не менее, применение роботов в сфере обслуживания сегодня не так распространенно. Оно находится на рубеже науки робототехники и пока ещё широко не применяется. В данной работе автор разрабатывает небольшую
часть робота, функционирующего в сфере обслуживания, главной целью которого
является общение с человеком. 

В частности цель работы - создать интерактивную систему слежения за человеческим лицом подобием головы робота, которая оборудована веб-камерами на месте глаз и серво-приводами, способными поворачивать её по двум осям. Вся система состоит из 3 модулей:
\begin{myItemize}
\item Нахождение местоположения и размеры лиц людей на изображении с веб-камеры
\item Выбор лица из найденных, за которым необходимо следовать
\item Вычисления вектора движения и само общение с серво-приводами
\end{myItemize}

Самой сложной из задач является поиск лица человека. В работе автор предлагает последовательный алгоритм поиска, который состоит из 3 подзадач, где результат предыдущей является источником данных для последующего:
\begin{myItemize}
\item Предварительная обработка и подготовка изображения;
\item Поиск, сегментация и кластеризация участков кадра, в которых высока вероятность обнаружения лица;
\item Применение искусственных нейронных сетей для окончательной классификации (лицо или нет) по нескольким представлениям данного изображения;
\end{myItemize}

TODO: Почему именно такой?

TODO: Про real-time сюда?

Подобная система может применяться в любых роботах, обладающих подобием головы.
Например: робот-консьерж в отеле, робот-официант или робот-домохозяйка. Это может упростить и улучшить
впечатление от общения человека с машиной.

TODO: Содержание глав

\chapter{Введение в предметную область}
\thispagestyle{fancy}

%(Про Computer Vision в целом.)
\section{Компьютерное зрение}

Основной частью данной работы является обработка изображений поступаемых с веб-камеры. Трансформация данных с видео камеры или из статичных изображений в новое представление или принимаемое решение называется - \emph{Компюьтерным Зрением} (\texttt{Computer Vision} или \texttt{CV})\citep{bradski2008learning} Т.е. программы и алгоритмы, которые в своей работе используют визуальную информацию - всё это компьютерное зрение.

Человеку, в силу своей зрительной природы, может показаться, что обработка визуальной информации - это очень просто. Но эта представление крайне ошибочно. Наш мозг разделяет визуальную информацию на множество каналов, в которых зашифрованы различные винды информации и посылает их мозг человека. В мозгу есть системы распределения внимания, в ходе работы корой, часть информации обрабатывается, а часть остаётся незамеченной. \citep{bradski2008learning} Даже сетчатка глаза - внутренняя поверхность глаза, заполненная светочуствительными клетками (колбочками и палочками), отвечает за предворительную обработку сигнала. На поверхности глаза около 130 миллионов светочувствительных элементов, а нервных окончаний идущих к мозгу в 100 раз меньше. Это говорит о том, что сетчатка сжимает информацию. В частности одной из её функций является \emph{обнаружение границ} \texttt{Edge detection}) \citep{RetinaOnWiki}

А что "видит" компьютер? Матрицу из чисел, представляющих собой интенсивность света в разных участках светочувствительной матрицы. При этом каждая ячейка этой матрицы кроме полезной информации содержит ещё и шум. И в этом наборе чисел надо найти машину или, например, идущего человека.

%Где и зачем применяется face detection?
Компьютерное зрение широко применяется в медицине, где оно помогает человеку анализировать визуальные данные. Например по снимку с МРТ\footnote{Магнитно-резонансная томография} указать на опухоль или другие патологии. Это задача называется \emph{распознаванием образов} (\texttt{Pttern recognition}). Распознавать можно так же другие объекты и образы. Например, система автоматического замера скорости на дорогах распознаёт автомобили и их регистрационные номера, а система безопасности распознаёт передвижение людей.
(TODO Тут картинку показать?)

%Про Machine Learning в целом?

\emph{Распознавание лица} (\texttt{Face recognition}) человека является одной из наиболее популярных задач в области компьютерного зрения. Она заключается в обнаружение и определении по изображению лица кому именно оно принадлежит. Решения этой проблемы применяются в системах безопасности (авторизация) и системах управления базами данных лиц людей. %TODO More Use cases
С быстрым развитием более развитых методов в этой области, распознавание лица человека, как средство авторизации, представляет из себя более дешёвое решение по сравнению с системами распознавания сетчатки глаза или отпечатка пальца. \citep{kumar2006efficient}

В популярной программе для работы с цифровыми фотографиями Picasa (\url{http://picasa.google.com/}) есть эффективная система распознавания, маркировки и каталогизирования найденных на фотографиях лиц людей. 
(TODO Тут картинку показать?)

\section{Распознавание и обнаружение лиц}

Но для того, что бы распознать лицо, сначала необходимо найти его место положение и границы на изображении. Эта задача называется \emph{обнаружение лица} (\texttt{Face detection}), что является частным случаем более общей проблемы \emph{обнаружение объекта} (\texttt{Object detection}). Почти все алгоритмы распознавания лиц в качестве входных данных используют изображение, содержащее только лицо, которое надо распознать. По-этому обнаружение лица есть предварительная и очень важная задача, которую надо выполнить, перед распознаванием. Следовательно от точности и быстроты определения местоположения лица зависит эффективность всей задачи по распознаванию. 

Но, обнаружение не обязательно должно вести к распознаванию. Обнаружение и слежение за лицом без определения пренадлежности его к конкретному человеку является основной задачей как данной работы, так и схожих по своей сути работ: \citep{capi2010vision}, \citep{luo2007face}, \citep{saxena2008real}.

Нахождение лица заключается в том, что бы по данному изображению определить, количество, место положения и размеры всех имеющихся лиц. \citep{liu2010automatically} (TODO Тут картинку показать?)

\section{Существующие методы обнаружения лиц}


Почти все современные подходы к решению задачи обнаружения лица, по мнению автора, содержат так или иначе 3 составляющих:

(TODO Не не. способы тут - 15, 14

\begin{description}

%\begin{description}
\item[Подготовка данных]\hfill \\
	Для успешного обнаружения лица поступившие данные необходимо подготовить для дальнейшей работы. Сюда входят и различные методы \emph{предварительные обработки} изображения в целом (т.н. \texttt{preprocessing}) и методы позволяющие уменьшить область поиска или \emph{область интереса} (\texttt{Range of interest} или \texttt{ROI}) для ускорения всего процесса обнаружения.
\item[Представление данных]\hfill \\ 
	Изначальное изображение в виде матрицы интенсивностей светочувствительного элемента камеры часто трансформируют в иные, более компактные отображения или \emph{представления} (\texttt{Representation}). Такие изменения чаще всего ведут к потере информации, но облегчают процесс классификации.
\item[Классификация]\hfill \\ 
	Само определение наличия или отсутствия в данном участке картинки лица человека. Методов классификации в принципе (не только лиц) на сегодняшний день существует огромное количество и все они применимы для данной задачи.
\end{description}

\subsection{Методы предварительной подготовки данных}
\subsubsection{Коррекция цвета и освещённости}
Для эффективной работы с изображением его необходимо подготовить.(О проблеме освещения в принципе)

Некоторые механизмы применяются для всего изображения в целом. Примером таких изменений могут служить коррекция контраста, интенсивности и баланса белого. Другие подготовительные процессы касаются изображений, которые поступают непосредственно в классификатор. Речь идёт о небольших частях изображения, которые получаются методов скольжения окна (\texttt{sliding window}) и маштабируемости, которые называют \emph{образцами} (об этом подробнее в (TODO)). Процессы коррекции контраста и баланса белого часто накладывают и на образцы, т.к. зачастую после исправления контрастности всего изображения, распределение интенсивности в образце остаётся узким.
(TODO Картинку с тусклым изображением и после наложения контраста?)

В работе \citep{rowley1998neural}, в частности, применяется два метода предворительной обработки образцов. Во-первых производится нелинейное \emph{уровновешивание гистограммы} (\texttt{Histogram equalization}), которое "растягивает" гистограмму, что компенсирует недостающие уровни интенсивности изображения, в результате чего изображение становиться более контрастным. Во-вторых производиться линейная компенсация интенсивности. Для этого линейная функция аппроксимирует общую яркость в каждой из частей образца, после чего она может быть вычтена из образца в случае сильной разницы в освещённости разных частей образца. (TODO Пример работы такого подхода) 

Похожий алгоритм нормализации интенсивности применяется и в работе \citep{lin2005face}. Перед классификацией каждый образец нормализуется по формулам \ref{eq:lin2005_1} и \ref{eq:lin2005_2}.

\begin{equation}
\label{eq:lin2005_1}
\centering
\bar{I}=\frac{1}{N}\sum_{i=1}^{N}I_{i}
\end{equation}
\begin{equation}
\label{eq:lin2005_2}
\centering
{I}'_i=(I_i-\bar{I})+128
\end{equation}

\begin{tabular}{p{3cm} c l}
где & $N$ & -- количество пикселей в образце\\
	& $\bar{I}$ & -- средняя интенсивность по всем пикселям образца\\
	& $I_i$ & -- интенсивность $i$'ого пикселя образца\\
	& ${I}'_i$ & -- нормализированная интенсивность\\
\end{tabular}

Таким образом если какие-то образцы были слишком тёмными или слишком светлыми они все становятся единообразно освещены, что упрощает процесс машинного обучения. (TODO Пример. Картинка)

В данной работе применяются схожие методы предварительной обработки образцов и изображения в целом. Подробнее в разделе ....(TODO)

\subsubsection{Сегментация по цвету кожного покрова}
Чем меньше изображние необходимо сканировать на наличие лица, тем меньше процессорного времени необходимо затрачивать, что ведёт к ускорению процесса обнаружение. Это можно достигнуть, например, уменьшением изображения, что приведёт к потере разрешающей способности данного подхода. Одним из способов позволяющих уменьшить площадь сканирования и одновременно не потерять разрешающей способности является эвртистическое знание о том, что все лица людей покрыти кожным покровом. Это можно использовать для нахождения областей изображения содержащих цвета схожие с цветами кожного покрова (\texttt{skin-color detection}), и в дальнейшем осуществлять сканировение только этих областей.

Среди методов обнаружения лиц основанных на поиске признаков (\texttt{feature based}) подходы использующие информацию о цвете кожи, как признак обнаружения, получают всё большую популярность. Цвет легко и быстро обрабатывать и он инвареантен к геометрическим особенностям образов лиц людей. К тому же опыт подсказывает, что цвет кожи человека имеет отчётливый характерный цвет, который легко узнаваем людьми. \citep{vezhnevets2003survey} Цвет кожи - это один из тех признаков, что не зависит от положения лица, частичной закрытости и контраста, по-этому именно этот метод часто используют для локализации лиц. \citep{ruangyam2009efficient}

Такой метод возможен благодаря тому факту, что различные цветовые вариации кожных покровов людей (даже среди представителей различных этнических групп) лежат в достаточно узком диапазоне и отличаются только яркостью (\texttt{Luminance}), в тож время цветность (\texttt{Chrominance}) практически не меняется. Цвет кожи лежит в основном в красной части цветового спектра и определяется цветом крови. А яркость определяется прозрачностью эпидермиса (верхнего слоя кожи), за что в свою очередь отвечает концентрация меланина. \citep{xu2006color} Подробнее в разделе ... (TODO)

\subsection{Методы представления изображения}
Одно и тоже изображение можно представить разными способами. Это нужно для того, что бы обучаемые классификаторы (см. \ref{classficators}) могли обучаться на различных характеристиках или признаках, имеющихся в различных представлениях. 



\subsubsection{Пиксельная интенсивность}
Самым простым и популярным представлением является информация об интенсивности в каждом пикселе изображения. Такой метод представления считается не имеющим потерь (\texttt{lossless}). Т.е. в нём присутствует  вся информация об оригинальном изображении.\citep{bojkovic2006face}
\subsubsection{Информация о контурах}
Любой метод обнаружения лица (да как и все другие алгоритмы в компьютерном зрении) должен быть стойким к таким вещам как поза голов, угол обзора, освещение и многим другим факторам. 
Для решения проблемы с освещением используется \emph{информация о контурах} (\texttt{edginess}) или \emph{градент изображения}. Одно и то же лицо под разными источниками света с точки зрения пиксельного представления совершенно отличны друг от друга, что делает проблемой для классификатора найти среди них нечто общее. С другой стороны информация о пограничных облостях в большей степени неизменна. \citep{ahmadyfard2008hierarchical}

Это представление, в отличая от пикселного представления, не содержит всей изначальной информации. Сохраняется информация лиш о пограничных контрастных зонах лица (глаза, брови, губы)
\subsubsection{Собственные лица}
\emph{Собственное лицо} (\texttt{eigenface}) - это набор \emph{собственных векторов} (\texttt{eigen vectors}), используемых для описания "стандартезированных  компанентов лица". Один образец лица принимается как точка в многомерном пространстве. Много образцов образуют некую область в этом пространстве. Задача заключается в нахождении \emph{главных компонент} (\texttt{principal components}) этой области, параметрами которых её можно описать используя значительно меньшее количество переменных, нежели для описания всех точек изначального пространства.\citep{turk1991eigenfaces}

Собственный вектор представляет из себя набор параметров, которыми можно описать лицо человека. Их же можно хранить например в базе данных, или использовать в качестве представления группы изображений. Этот набор параметров является своего рода выжемкой из многих тренировочных лиц и может использоваться для дальнейшей классификации.

В работе \citep{tsai2006face} автор для обнаружения лиц использует собственные лица для поиска кондидатов областей с лицом, а нейронная сеть проводит конечную валидацию.
\subsubsection{Характеристики типа Хаара}
\emph{Характеристики типа Хаара} (\texttt{Haar-like features})  представляют изображения при использовании платформы по обнаружению объектов Виола-Джонса (\texttt{Viola-Jones object detection framework}). \citep{viola2001rapid}
Эта метод и применяемое в нём представление является одним из самых быстрых на сегодняшний день методов обнаружения объектов.
%Мало, может ещё чего. Подробнее?
\subsubsection{Использование нескольких представлений}
Многие решения обнаружения лица используют сразу несколько представлений. Это позволяет использовать более широкий спектр характеристик, что делает классификатор более чувствительным к особенностям конкретного вида объектам (в данном случае к лицам).

Так в работе \citep{bojkovic2006face} используются три представления: пиксельное, коэфициенты собственных лиц и профильные коэфициенты. В работе \citep{ahmadyfard2008hierarchical} авторы используют две представления: пиксельное и информацию о контурах. По их словам "объединение информации об интенсивности и о контурах даёт более описательные характеристики для представления изображения с лицом".

\subsection{Задача классификации}
\label{classficators}
Задача \emph{классификации} - это проблема определение класса из всех возможных, к которому относиться классифицируемый объект или наблюдение. Классификация тесно связана с \emph{машинным обучением} (\texttt{Machine Learning} или \texttt{ML}), задачей которого является превращение данных в информацию \citep{bradski2008learning}

Задача обнаружения лица является классическим примером задачи классификации, где есть всего два класса "лицо" или "не лицо". Классификатор должен по данному ему изображению должен уметь определить к какому классу из двух оно относиться.
\subsubsection{SVM}
\subsubsection{PCA}
\subsubsection{ИНС}
1. 13.


* Обзор методов и решений. (способы face detection’a) (для каждого подпункт?) (С ROI, с цветом одежды, с отделением фона, с выделением движ. объектов)

* целые системы подобно реализуемой (голова робота и всё такое)

* про real-time

\chapter{Предлагаемый метод решения}
\thispagestyle{fancy}

Краткое описание всей системы. Диаграмма.

Модульность. Какие-то части могут быть реализованы по разному - но сами модули такие как тут.

используемые програмные библиотеки: OpenCV, PyBrain, ...

\section{Модуль нахождения лица}

\subsection{Нормализация контраста и баланс белого}
(предобработка)

Описание алгоритма.

Много примеров, гистограмм, псевдокод. (без особых результирующих картинок)

\subsection{Поиск зон с цветом кожного прокрова}
Общая информация о проблеме.

- зачем применяют 

	--про поиск картинок для взрослых
	
	--для face detetection
\subsubsection{Возможные пути решения}
Уже существующие различные методы нахождения цвета кожного покрова. Описание тут.
\subsubsection{Проблема выбора цветового пространства}
Про цветовые пространства. Про информативность каждого из них.
\subsubsection{Метод статического диапозона}
В работе реализуется он. Почему? (просто, быстро, достаточно эффективный)
Описание метода, псевдокод?

Сравнение двух реализованных моделей.

\subsection{Выделение и объединение областей с цветом кожного покрова}
Общии слова переходного характера. 
\subsubsection{Выделение найденных областей}
Описание алгоритма (сжатие, расширение) -  избавление от шума, более адекватные замкнутые области.
\subsubsection{Кластеризация}
Обоснование необходимости. Зачем объеденять.	

Почему обычный k-mean не подходит? примеры.

Описание метода кластеризации через минимальное оставное дерево. Что такое оставное дерево?

Примеры мест (ситуаций), где это необходимо. Где лицо состоит из нескольких небольших участков и полезно объеденять.

Псевдокод, диаграммы процесса.
\subsection{Фильтрация по пропорциям и заполненности}
Описание возможного постпроцессинга для отфильтровывания неподходящих участков.
\subsection{Классификация}
Описание проблеммы классификации в целом.
Опять о том какие методы бывают. О том, что сейчас применяют чаще.
\subsubsection{Выбор метода ИНС для классификации}
Почему выбрал ann? (real-time, простота понимания и использования)

Как это делают другие?
\subsubsection{Описание сети}
Несколько сетей для разных представлений. B/w, Edges

О проблемах недофитинга и overfit'инга.

Описание структуры ИНС. Почему именно такая.

bias'ы, преждевременная остановка, ...

Код с сосзданием сети.
\subsubsection{Обучение сети}
Первый этап.

Проблема и важность выбора примеров для обучения.

Применяемые базы лиц, усреднённые лица, возможная рамочка

Проблема выбора "не лиц".

Описание подготовки выборок для тренировки и тестирования.

Код PyBrain по тренеровки сети.
\subsubsection{Применение сети}
Сохранение и загрузка обученной сети.

Понятие порога.

sliding window алгоритм. диаграмы, код.

Кластеризация всех найденных лиц в группы, что бы отсечь случайные Flase positives. Overlap'ы и всё такое.
\section{Выбор цели для слежения}
найти наибольшее лицо

искать ближайшее к тому, за которым уже следим
\section{Механическая часть}
Работа с сервоприводами
\subsection{описание установки для демонстрации}
arduino,
сервоприводы,
камеры
\subsection{Подсчёт вектора движения}
\subsection{Arduino}
коммуникация c PC

листинги кода, диаграммы (этого нет =/ )


\chapter{Результаты работы (Испытания?)}
%\addcontentsline{toc}{chapter}{Результаты работы (Испытания?)}
\thispagestyle{fancy}

\section{Автоконтраст и баланс белого}
Когда работает? А когда не очень? примеры и того и того.

Возможные пути решения.
\section{Поиск зон с кожным покровом}
Когда работает? А когда не очень? примеры и того и того.

Пути решение. (Другой метод, выбор более узкой области диапазонов. - пример возможного приложения для сбора образцов)

О том что хорошо, что больше false negative, чем false positive
\section{Объединение областей}
Примеры хорошой и плохой работы.

Как можно улучшить. (выбор другого алгоритма выбора цвета кожи, подгонка параметров кластеризации)
\section{(Результаты) работа с ИНС}
\subsection{Различные представления}
Почему представление с пограничными областями не работает. Усреднённые морды где видно проблему. Как-то улучшить алгоритм выявления пограничных областей? Какие-то другие представления?
\subsection{(Результаты) обучения и тестирования}
Цифры, проценты результативности на тестовых данных. False positive, False negative. Примеры неузнанных лиц, примеры узнанных не лиц. Усреднённые нелица, усреднённые ненайденные лица.
Как можно улучшить?

- Правильная структура сети, выделяющая характеристики (features)

- Более тщательный подбор примеров (глаза на одном месте, одна ореинтация)

- Разные классы для разных поз (направление взгляда - прямо, вверх, вниз, вправо, влево)

- Икрементальный процесс обучения (где все falses из тестового набора добавляются обратно в набор обучения)
\section{Выбор лица и arduino}
Так и не успел закончить эту часть. Что писать в результатах пока не знаю.
\section{Испытание всей системы}
результаты испытаний.

небольшие ошибка на всех уровнях в итоге дают неудовлетворительный результат. улучшать необходимо каждый из элементов в отдельности.

- о проблеме 2-7 процентах на 97 000 примерах с одного кадра. Много false positive. Сложно настроить правильные порог.

- о проблеме со скорость. Решение - оптимизация и использование psyco.

\chapter*{Заключение и выводы}
\addcontentsline{toc}{chapter}{Заключение и выводы}
\thispagestyle{fancy}


\appendix
\chapter{Приложение. Отчёт по курсовой практике}
%\pagebreak
%-

%\clearpage
%\pagebreak
%-

%\clearpage
%\pagebreak
%-

%\clearpage
%\pagebreak
%-
%\clearpage


%\clearpage

\addcontentsline{toc}{chapter}{Литература}

\bibliographystyle{plainnat}
\bibliography{biblio}

\end{document}  